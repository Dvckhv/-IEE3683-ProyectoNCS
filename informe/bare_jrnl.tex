\documentclass[journal]{IEEEtran}

\ifCLASSINFOpdf
  \usepackage[pdftex]{graphicx}
\else
\fi

\usepackage{amsmath}
\usepackage{amssymb}
\usepackage{cite}
\usepackage{algorithm}
\usepackage{algorithmic}
\usepackage{booktabs}
\usepackage{url}

\hyphenation{op-tical net-works semi-conduc-tor}

\begin{document}

\title{Extended Comparative Analysis of Kinematic and Dynamic Controllers in NOMA-V2V Platoon Systems with Jerk-Constrained Motion}

\author{Jorge Apud, Patricio Hinostroza%
\thanks{Los autores son estudiantes de la Pontificia Universidad Católica de Chile (emails: patricio.hinostrozav@uc.cl, japud@uc.cl).}%
}

\markboth{IEE3683--Networked Control Systems}{}

\maketitle

\begin{abstract}
Los sistemas de control vehicular cooperativo han emergido como una solución esencial para mejorar eficiencia, seguridad y capacidad vial en entornos de transporte inteligente. En \cite{He2023} se introduce un esquema conjunto de asignación de recursos NOMA y control distribuido orientado a platooning, modelando explícitamente la relación entre error de espaciamiento y demanda de comunicación. Sin embargo, dicho marco utiliza un modelo cinemático idealizado, donde la aceleración es directamente manipulada por el controlador, ignorando la dinámica real del actuador, límites de fuerza y restricciones de \textit{jerk}. En esta investigación se replica rigurosamente el marco original y se propone una extensión físicamente realizable basada en un modelo dinámico fuerza–masa–\textit{jerk}. Los resultados muestran que el modelo cinemático subestima el error real y genera perfiles de aceleración incompatibles con estándares de confort, mientras que el modelo dinámico corrige esta brecha a costa de un moderado aumento en demanda de comunicación. El trabajo contribuye al entendimiento integrado control–comunicación en redes NOMA-V2V con dinámica vehicular realista.
\end{abstract}

\begin{IEEEkeywords}
NOMA-V2V, Platooning, Vehicular Dynamics, MPC, Jerk-Limited Control, Distributed Control, Resource Allocation.
\end{IEEEkeywords}

\section{Introducción}

\IEEEPARstart{E}{l} control cooperativo de formaciones vehiculares (\textit{platooning}) es un componente esencial de los Sistemas de Transporte Inteligente (ITS). Al mantener distancias cortas y velocidades coordinadas, los vehículos reducen consumo energético, aumentan capacidad vial y mejoran seguridad \cite{Jia2016}. Estos beneficios dependen críticamente de la calidad de comunicación V2V y de la disponibilidad de recursos espectrales \cite{Gyawali2021}.

La proliferación de dispositivos conectados ha generado una presión creciente sobre las bandas de comunicación, motivando la adopción de técnicas no ortogonales como NOMA, que permiten multiplexar múltiples usuarios en el dominio de potencia \cite{Ding2017, Saito2013}. Entre los trabajos recientes, He y Leung \cite{He2023} presentan un marco conjunto de control distribuido y asignación de recursos NOMA que modela de manera explícita la relación entre error de espaciamiento, capacidad de corrección y frecuencia de comunicación requerida.

No obstante, la gran mayoría de trabajos en control de platooning, incluyendo \cite{He2023, Mei2018, Zhang2022}, utilizan modelos cinemáticos de segundo orden donde la aceleración es manipulada directamente por el controlador. Estos modelos ignoran la dinámica real del actuador y la presencia de límites físicos como fuerza máxima, tiempo de respuesta y restricciones de \textit{jerk}. Estudios recientes en conducción autónoma \cite{Czarnecki2018, Rajamani2012} indican que violar estos límites puede generar comportamientos no confortables o incluso peligrosos.

El presente trabajo replica fielmente el algoritmo planteado en \cite{He2023} y extiende su formulación mediante un modelo dinámico $F=ma$ con entrada $u=\dot{F}$, imponiendo restricciones explícitas de \textit{jerk}. El objetivo es comparar cuantitativamente ambos modelos bajo igual estructura NOMA-V2V, respondiendo:

\begin{itemize}
\item ¿Qué costo adicional de comunicación implica un modelo dinámico físicamente realizable?
\item ¿Se mantiene la estabilidad de la cuerda al limitar el \textit{jerk}?
\item ¿Cuánto difieren las trayectorias respecto al modelo original?
\end{itemize}

Este artículo constituye, hasta donde sabemos, el primer análisis comparativo que integra NOMA-V2V, control distribuido y dinámica vehicular realista.

\section{Marco General del Sistema}

\subsection{Arquitectura de Comunicación NOMA-V2V}

El sistema utiliza comunicación V2V entre cada vehículo y su predecesor. Cada período de control está compuesto por $T$ \textit{timeslots}. Al igual que en \cite{He2023}, hasta $O_{\max}=2$ vehículos pueden transmitir concurrentemente mediante NOMA. El receptor aplica Cancelación Sucesiva de Interferencia (SIC) siguiendo el \textbf{Teorema 1}  enunciado en \cite{he2023}.
\begin{quote}
El orden óptimo de decodificación es el orden ascendente en ganancia de canal de interferencia.
\end{quote}

Este orden minimiza potencia total al garantizar que los usuarios que sufren más interferencia decodifiquen primero.

La SINR del vehículo $i$ en el instante $t$ queda:
\begin{equation}
\Gamma_i^t=\frac{o_i^t p_i^t h_{i,i-1}^t}{I_i^t+N_0},
\end{equation}
y debe cumplir $\Gamma_i^t \ge \Gamma_{\text{th}}$.

\subsection{Demanda de Comunicación Basada en Error}

Siguiendo \cite{He2023}, la demanda de comunicación está determinada por el error de espaciamiento:
\[
e_{i,p}^t = p_{i-1}^t - p_i^t - d - l.
\]
Si $|e_{i,p}^t|\le e_{\text{th}}$, el vehículo no requiere comunicación; de lo contrario, el número de comunicaciones necesarias por período viene dado por:
\begin{equation}
f_i=\min\left\{
\frac{|e_{i,p}|}{\frac{1}{T} \cdot \frac{1}{2}u_{\max}(T\Delta t)^2},
\frac{O_{\max} T}{N}
\right\}.
\end{equation}

\section{Modelos Vehiculares Considerados}

\subsection{Modelo A — Cinemático (He \textit{et al.}, 2023)}

El modelo idealizado considera:
\begin{align}
v_{i}^{t+1} &= v_{i}^{t} + u_{i}^{t}\Delta t,\\
p_{i}^{t+1} &= p_{i}^{t} + v_i^t\Delta t + \frac{1}{2}u_i^t(\Delta t)^2.
\end{align}
La entrada $u_i^t$ es directamente la aceleración.

\subsection{Limitaciones del modelo cinemático}

Este modelo ignora:
\begin{itemize}
\item dinámica del actuador (retardo),
\item fuerza máxima disponible,
\item limitaciones de torque en motores eléctricos o ICE,
\item límites de \textit{jerk} para confort \cite{Czarnecki2018},
\item saturación del sistema de frenado.
\end{itemize}

\subsection{Modelo B — Dinámico Propuesto (Limitado por Jerk)}

Se introduce la fuerza $F_i$ como estado adicional:
\begin{align}
F_{i}^{t+1} &= F_i^t + u_i^t \Delta t, \\
a_i^{t+1} &= F_i^{t+1}/m,\\
v_i^{t+1} &= v_i^t + a_i^{t+1}\Delta t,\\
p_i^{t+1} &= p_i^t + v_i^t\Delta t + \tfrac{1}{2}a_i^{t+1}(\Delta t)^2.
\end{align}
La entrada $u_i^t$ corresponde al cambio de fuerza, equivalente al \textit{jerk} escalado:
\[
jerk \approx \frac{u_i^t}{m}.
\]

Restricciones:
\begin{align}
F_{\min}\le F_i \le F_{\max},\\
u_{\min}\le u_i^t \le u_{\max},\\
|jerk| < 0.9\text{ m/s}^3.
\end{align}

Este modelo es estándar en control automotriz y congruente con \cite{Rajamani2012, Paden2016}.

\section{Control Predictivo Distribuido (QP)}

Cada vehículo minimiza:
\begin{equation}
\min_{u_i^t} (e_{i,est}^{t+1})^T W (e_{i,est}^{t+1}),
\end{equation}
con restricciones en $u_i^t$.

\subsection{Modelo dinámico: nueva matriz predictiva}

El error futuro depende de $u_i^t$ mediante
\[
e_{p}^{t+1} \approx e_p^t + e_v^t \Delta t + \frac{\Delta t^3}{2m}u_i^t.
\]
La matriz resultante es:
\[
A = 
\begin{bmatrix}
-\frac{\Delta t^3}{2m}\\
-\frac{\Delta t^2}{m}\\
-\frac{\Delta t}{m}
\end{bmatrix},
\qquad
H = 2A^T W A.
\]

\section{Simulaciones}

\subsection{Parámetros de simulación}

\begin{itemize}
\item $N=12$ vehículos, $d=10$ m, $l=5$ m.
\item $\Delta t=10$ ms, $T=10$.
\item $m=1500$ kg, $jerk_{\max}=0.9$ m/s$^3$.
\item $p_{\max}=35$ dBm.
\item Escenario de frenado y aceleración usado por \cite{He2023}.
\end{itemize}

\subsection{Resultados}

\subsubsection{Perfil de aceleración}
El modelo cinemático utiliza $u_{\max}$ agresivamente, generando discontinuidades bruscas en $a(t)$.  
El modelo dinámico restringe $u$, produciendo transitorios suaves.

\subsubsection{Estabilidad de la cuerda}
Se observó:
\[
\text{overshoot}_{\text{cin}} \approx 0.03\text{ m}, \qquad 
\text{overshoot}_{\text{dyn}} \approx 0.10\text{ m},
\]
pero el dinámico mantiene gradientes continuos en aceleración.

\subsubsection{Consumo de recursos de comunicación}
El modelo dinámico requiere \textbf{15–18\% más transmisiones}, debido a microcorrecciones frecuentes al limitar el \textit{jerk}.

\section{Conclusiones}

Se replicó la arquitectura NOMA–control distribuido de He \textit{et al.} (2023) y se mostró que el modelo cinemático produce perfiles físicamente inviables. El modelo dinámico fuerza–masa–\textit{jerk} propuesto es compatible con estándares de confort humano y con la física del actuador. La penalización en recursos de comunicación es moderada, demostrando la viabilidad del enfoque en redes V2V modernas.

\vspace{6pt}

\appendices
\section{Matrices del Controlador Dinámico}
\begin{verbatim}
A_p = -0.5 * (dt^3 / m);
A_v = - (dt^2 / m);
A_a = - (dt / m);
H = 2 * (A' * W * A);
\end{verbatim}

\begin{thebibliography}{14}

\bibitem{He2023}
Q. He and K.-C. Leung, ``Joint Platoon Control and Resource Allocation for NOMA-V2V Communication System,'' \emph{IEEE Trans. Veh. Technol.}, 2023.

\bibitem{Jia2016}
D. Jia et al., ``A Survey on Platoon-Based Vehicular Cyber-Physical Systems,'' \emph{IEEE Communications Surveys}, 2016.

\bibitem{Gyawali2021}
S. Gyawali et al., ``Challenges and Solutions for Cellular-Based V2X Communications,'' \emph{IEEE Communications Surveys}, 2021.

\bibitem{Ding2017}
Z. Ding et al., ``A Survey on Non-Orthogonal Multiple Access for 5G Networks,'' \emph{IEEE JSAC}, 2017.

\bibitem{Saito2013}
Y. Saito et al., ``NOMA for Cellular Future Radio Access,'' in \emph{Proc. IEEE VTC}, 2013.

\bibitem{Czarnecki2018}
K. Czarnecki, ``Automated Driving System (ADS) Quality Requirements – Comfort Analysis,'' Univ. of Waterloo, 2018.

\bibitem{Mei2018}
J. Mei et al., ``Joint Radio Resource Allocation and Control for Vehicle Platooning,'' \emph{IEEE TVT}, 2018.

\bibitem{Zhang2022}
D. Zhang et al., ``Joint Resource Allocation and Control for Resource-Constrained Vehicle Platooning,'' in \emph{Proc. IEEE GLOBECOM}, 2022.

\bibitem{Ge2022}
X. Ge et al., ``Scalable and Resilient Platooning Control,'' \emph{IEEE TVT}, 2022.

\bibitem{Noor2022}
M. Noor-A-Rahim et al., ``Resource Allocation in Vehicular Networks,'' \emph{IEEE T-ITS}, 2022.

\bibitem{Rajamani2012}
R. Rajamani, \emph{Vehicle Dynamics and Control}, Springer, 2012.

\bibitem{Paden2016}
B. Paden et al., ``A Survey of Motion Planning for Automated Vehicles,'' \emph{IEEE T-ITS}, 2016.

\bibitem{Lofberg2004}
J. Lofberg, ``YALMIP: A Toolbox for Optimization in MATLAB,'' in \emph{IEEE CACSD}, 2004.

\end{thebibliography}

\end{document}
