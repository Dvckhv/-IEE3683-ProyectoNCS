\documentclass[journal]{IEEEtran}

% *** GRAPHICS RELATED PACKAGES ***
\ifCLASSINFOpdf
  \usepackage[pdftex]{graphicx}
\else
\fi

\usepackage{amsmath}
\usepackage{amssymb}
\usepackage{cite}
\usepackage{algorithm}
\usepackage{algorithmic}
\usepackage{booktabs}
\usepackage{url}
\hyphenation{op-tical net-works semi-conduc-tor}

\begin{document}

\title{Extended Comparative Analysis of Kinematic and Dynamic Controllers in NOMA-V2V Platoon Systems with Jerk-Constrained Motion}

\author{Jorge Apud, Patricio Hinostroza%
\thanks{Los autores son estudiantes de la Pontificia Universidad Católica de Chile (emails: patricio.hinostrozav@uc.cl, japud@uc.cl).}%
}

\markboth{IEE3683--Networked Control Systems}{}

\maketitle

\begin{abstract}
Los sistemas de control vehicular cooperativo han emergido como una solución esencial para mejorar la eficiencia, seguridad y capacidad vial en entornos de transporte inteligente. En \cite{He2023} se introduce un esquema conjunto de asignación de recursos NOMA y control distribuido orientado a \textit{platooning}, modelando explícitamente la relación entre el error de espaciamiento y la demanda de comunicación. Sin embargo, dicho marco utiliza un modelo cinemático idealizado, donde la aceleración es directamente manipulada por el controlador, ignorando la dinámica real del actuador, los límites de fuerza y las restricciones de \textit{jerk} (tasa de cambio de aceleración). En esta investigación se replica rigurosamente el marco original y se propone una extensión físicamente realizable basada en un modelo dinámico de tercer orden (fuerza–masa–\textit{jerk}). Los resultados comparativos muestran que el modelo cinemático subestima el error de rastreo real y genera perfiles de aceleración incompatibles con los estándares de confort de pasajeros, mientras que el modelo dinámico corrige esta brecha a costa de un moderado aumento en la demanda de comunicación. El trabajo contribuye al entendimiento integrado del co-diseño control–comunicación en redes NOMA-V2V bajo restricciones dinámicas realistas.
\end{abstract}

\begin{IEEEkeywords}
NOMA-V2V, Platooning, Vehicular Dynamics, MPC, Jerk-Limited Control, Distributed Control, Resource Allocation.
\end{IEEEkeywords}

\section{Introducción}


\IEEEPARstart{E}{l} control cooperativo de formaciones vehiculares (\textit{platooning}) constituye un pilar fundamental de los Sistemas de Transporte Inteligente (ITS) modernos. Mediante la mantención de distancias reducidas entre vehículos y la coordinación de velocidades, se reduce el consumo energético, aumenta la capacidad vial, mejora seguridad, reducen trafico, entre otros \cite{Jia2016},\cite{Noor2022}.

Para garantizar la estabilidad del pelotón, \textit{string stability}, es imperativo contar con un controlador eficaz que opere sobre una red de comunicación Vehículo-a-Vehículo (V2V) de alta fiabilidad \cite{Jia2016}. Esto plantea una dependencia crítica entre el desempeño del control y la disponibilidad de recursos espectrales \cite{Gyawali2021}. Una red deficiente, caracterizada por latencias estocásticas o pérdida de paquetes, puede degradar la calidad del control, llevando a oscilaciones peligrosas o incluso colisiones.

\subsection{Desafíos en la Gestión de Recursos}
La proliferación exponencial de dispositivos conectados en el Internet de las Cosas (IoT) ha generado una presión creciente sobre las bandas de comunicación limitadas. Como se señala en \cite{Noor2022}, sin una gestión de recursos rigurosa, la comunicación V2V se vuelve propensa a interferencias y congestión, haciendo difícil satisfacer los estrictos requisitos de latenciay tasa de actualización necesarios para el control de pelotones.

La importancia de la gestión de potencia en canales con interferencia fue establecida tempranamente por Kandukuri y Boyd \cite{Kandukuri2002}, quienes demostraron que minimizar la probabilidad de interrupción (\textit{outage probability}) es equivalente a maximizar el margen de la relación señal-a-interferencia (SIR). En el contexto vehicular, se extendió este análisis en \cite{Han2022}, evidenciando que las interrupciones de comunicación obligan a aumentar el tiempo de \textit{headway} para mantener la seguridad, reduciendo así la eficiencia del tráfico.

En este contexto, diversos estudios han explorado la integración de técnicas avanzadas de comunicación con esquemas de control distribuido. En \cite{Ge2022} abordaron el control resiliente y escalable de vehículos conectados, proponiendo estrategias para mitigar incertidumbres en la topología de la red. Sin embargo, su enfoque asume a menudo condiciones de canal ideales.

Desde la perspectiva de la asignación de recursos, en \cite{Mei2018} se aborda la optimización conjunta de recursos de radio y control en redes LTE-V2V. Su enfoque se basó en tecnologías de Acceso Múltiple Ortogonal (OMA), donde los recursos de tiempo y frecuencia se dividen exclusivamente entre usuarios para evitar interferencia. Si bien esto simplifica el diseño, Posteriormente se demostró que los esquemas OMA sufren de una baja eficiencia espectral en escenarios de alta densidad vehicular, limitando la escalabilidad del sistema \cite{Zhang2022}.

Para superar las limitaciones de OMA, trabajos recientes \cite{He2023} exploran la adopción de técnicas de Acceso Múltiple No Ortogonal (NOMA). NOMA permite multiplexar múltiples usuarios en el mismo recurso de tiempo-frecuencia diferenciándolos en el dominio de la potencia \cite{Ding2017, Saito2013}.Así, \cite{He2023} presenta un marco conjunto de control distribuido y asignación de recursos NOMA que modela de manera explícita la relación entre el error de espaciamiento actual y la frecuencia de comunicación requerida, logrando una eficiencia superior a los esquemas OMA tradicionales.

No obstante, la gran mayoría de trabajos en co-diseño de control y comunicaciones, incluyendo \cite{He2023, Mei2018, Zhang2022}, utilizan modelos cinemáticos de segundo orden simplificados. En estos modelos, la aceleración se considera una entrada de control directa que puede cambiar instantáneamente. Esta asunción ignora la dinámica real del actuador (motor/frenos), la masa del vehículo y, crucialmente, las restricciones de \textit{jerk} (tasa de cambio de aceleración).

Estudios en dinámica vehicular \cite{Rajamani2012} y análisis de calidad de conducción autónoma \cite{Czarnecki2018} enfatizan que violar los límites de \textit{jerk} (típicamente $<0.9$ m/s$^3$ para confort) genera comportamientos inaceptables para los pasajeros y físicamente irrealizables por los actuadores mecánicos, limitando la aplicabilidad de estos esquemas de control.

El presente trabajo replica fielmente el algoritmo NOMA-V2V planteado en \cite{He2023} y extiende su formulación mediante un modelo dinámico de tercer orden ($F=ma$, $\dot{F}=u$), imponiendo restricciones explícitas de \textit{jerk}. El objetivo es comparar cuantitativamente ambos modelos bajo la misma arquitectura de red, respondiendo:
\begin{itemize}
\item ¿Qué costo adicional de comunicación implica adherirse a un modelo dinámico físicamente realizable?
\item ¿Cómo afecta la limitación de \textit{jerk} a la convergencia del error de espaciamiento?
\item ¿Qué diferencias emergen en los perfiles de aceleración y velocidad resultantes?
\item ¿Es NOMA una solución robusta para absorber la carga de comunicación adicional impuesta por modelos más realistas?
\end{itemize}

\section{Marco General del Sistema}

\subsection{Arquitectura de Comunicación NOMA-V2V}
El sistema utiliza comunicación V2V entre cada vehículo y su predecesor en un esquema TDMA. Siguiendo a \cite{He2023}, Solo  $O_{\max}=2$ vehículos pueden transmitir en el mismo \textit{timeslot} utilizando NOMA de dominio de potencia.

El receptor aplica Cancelación Sucesiva de Interferencia (SIC). Para ello, se utiliza el \textbf{Teorema 1} de \cite{He2023}, el cual demuestra que, almenos para una comunicación entre 2 vehiculos, el orden óptimo de decodificación es el orden ascendente de la ganancia del canal de interferencia. Esto minimiza la potencia total requerida al decodificar primero a los usuarios que sufren mayor interferencia externa.

La condición de éxito para el vehículo $i$ en el instante $t$ es:
\begin{equation}
\Gamma_i^t=\frac{o_i^t p_i^t h_{i,i-1}^t}{I_i^t+N_0} \ge \Gamma_{\text{th}}
\end{equation}
donde $h$ es la ganancia del canal (incluyendo \textit{path loss} y desvanecimiento Rayleigh) y $I$ es la interferencia NOMA intra-slot.

\subsection{Agendamiento Basado en Eventos}
La demanda de comunicación no es estática. Se calcula en función del error de espaciamiento absoluto:
\[
e_{i,p}^t = p_{i-1}^t - p_i^t - d - l.
\]
Si $|e_{i,p}^t| > e_{\text{th}}$, el vehículo solicita recursos. La frecuencia de asignación $f_i$ se escala proporcionalmente a la magnitud del error relativo a la capacidad máxima de corrección del vehículo en un periodo $T$.

\section{Modelos Vehiculares Comparados}

Se analizan dos formulaciones para el diseño del controlador predictivo.

\subsection{Modelo A — Cinemático (Baseline \cite{He2023})}
Este modelo asume que el sistema de control vehicular puede imponer una aceleración objetivo instantáneamente.
\begin{align}
v_{i}^{t+1} &= v_{i}^{t} + u_{i}^{t}\Delta t,\\
p_{i}^{t+1} &= p_{i}^{t} + v_i^t\Delta t + \frac{1}{2}u_i^t(\Delta t)^2.
\end{align}
Aquí, la entrada de control es $u_i^t \equiv a_i^t$. Las restricciones son simplemente $a_{\min} \le u_i^t \le a_{\max}$.

\textit{Crítica:} Este modelo permite cambios de aceleración tipo escalón (step), lo que implica un \textit{jerk} infinito teórico y fuerzas instantáneas imposibles de generar por un motor real.

\subsection{Modelo B — Dinámico Propuesto (Limitado por Jerk)}
Para capturar la inercia y el confort, se introduce la fuerza longitudinal $F_i$ como variable de estado.
\begin{align}
F_{i}^{t+1} &= F_i^t + u_i^t \Delta t, \\
a_i^{t+1} &= F_i^{t+1}/m,\\
v_i^{t+1} &= v_i^t + a_i^{t+1}\Delta t,\\
p_i^{t+1} &= p_i^t + v_i^t\Delta t + \tfrac{1}{2}a_i^{t+1}(\Delta t)^2.
\end{align}
La nueva entrada de control es $u_i^t = \dot{F}$, que representa la tasa de cambio de fuerza. Dado que $a = F/m$, esto controla directamente el \textit{jerk}:
\[
jerk \approx \frac{u_i^t}{m}.
\]
Las restricciones se amplían para incluir límites de confort según \cite{Czarnecki2018}:
\begin{align}
F_{\min} &\le F_i \le F_{\max} \quad (\text{Límite Motor/Freno})\\
|u_i^t|/m &\le 0.9\text{ m/s}^3 \quad (\text{Límite Confort})
\end{align}

\section{Control Predictivo Distribuido (QP)}

Cada vehículo resuelve un problema de optimización local en cada paso de tiempo para minimizar el error de seguimiento futuro proyectado.
\begin{equation}
\min_{u_i^t} (e_{i,est}^{t+1})^T W (e_{i,est}^{t+1})
\end{equation}

Para el **Modelo A**, la proyección es cuadrática estándar. Para el **Modelo B**, derivamos una nueva relación de predicción. El error futuro $e(t+1)$ depende cúbicamente del tiempo respecto a la entrada de control $u$ (jerk):
\[
e_{p}^{t+1} \approx e_p^t + e_v^t \Delta t + \frac{1}{2}a^t \Delta t^2 + \frac{\Delta t^3}{6m}u_i^t.
\]
(Nota: En la implementación discreta simplificada, usamos los coeficientes matriciales derivados a continuación).

La matriz $\mathbf{A}$ que relaciona la entrada $u$ con el estado futuro $[p, v, a]^T$ es:
\[
\mathbf{A}_{dyn} = 
\begin{bmatrix}
-\frac{\Delta t^3}{2m}\\
-\frac{\Delta t^2}{m}\\
-\frac{\Delta t}{m}
\end{bmatrix}
\]
Esto penaliza implícitamente los cambios rápidos, actuando como un filtro paso bajo mecánico.

\section{Simulaciones y Resultados}

\subsection{Configuración del Experimento}
Se simula un pelotón de $N=12$ vehículos en MATLAB.
\begin{itemize}
\item Parámetros Físicos: $\Delta t=10$ ms, $m=1500$ kg.
\item Parámetros de Red: $BW=180$ kHz, $N_0=-174$ dBm/Hz, Potencia máx = 35 dBm.
\item Escenario: El líder ejecuta un perfil de velocidad variable (frenado de 20 a 15 m/s y posterior recuperación), introduciendo una perturbación que se propaga hacia atrás.
\end{itemize}

\subsection{Análisis Comparativo}

\subsubsection{Perfiles de Movimiento}
El modelo cinemático (A) utiliza la aceleración máxima permitida de forma inmediata, generando perfiles de velocidad con "codos" agudos. Si bien esto minimiza el error matemático rápidamente, requiere una respuesta de motor instantánea inexistente.
El modelo dinámico (B) produce curvas de velocidad sigmoidales (suaves). El \textit{jerk} se mantiene acotado, lo que resulta en una experiencia de conducción confortable y segura para la carga o pasajeros.

\subsubsection{Estabilidad y Error}
Se observó un sobrepaso (\textit{overshoot}) en el error de espaciamiento mayor en el modelo dinámico ($\approx 0.10$ m) comparado con el cinemático ($\approx 0.03$ m). Esto es una consecuencia física inevitable: la inercia impide corregir el error instantáneamente. Sin embargo, ambos modelos mantienen la estabilidad asintótica, convergiendo a cero error en estado estacionario.

\subsubsection{Costo en Comunicaciones}
Una hallazgo clave es el impacto en la red. El modelo dinámico requiere aproximadamente un **15–18\% más de transmisiones** totales.
\textit{Razón:} El modelo cinemático corrige el error en pocos pasos agresivos y luego entra en reposo (error $< e_{th}$). El modelo dinámico, al suavizar la corrección, tarda más tiempo en reducir el error por debajo del umbral $e_{th}$, manteniendo activa la solicitud de slots de comunicación por más periodos.

\section{Conclusiones}

Este trabajo replicó la arquitectura de control NOMA-V2V de He \textit{et al.} y demostró que los modelos cinemáticos utilizados en la literatura subestiman los requisitos de comunicación y la realidad física del control.

La propuesta de un modelo dinámico limitado por \textit{jerk} evidencia que:
1. La estabilidad es alcanzable bajo restricciones de confort, pero con una convergencia más lenta.
2. Existe un costo oculto en la eficiencia espectral: sistemas más realistas y suaves requieren una mayor tasa de actualización de datos para gestionar la inercia de forma precisa.
3. NOMA es una tecnología habilitadora robusta que permite absorber este exceso de carga de comunicación sin saturar el canal, validando su uso en VCPS de alta fidelidad.

\appendices
\section{Matrices del Controlador Dinámico}
Implementación MATLAB de las matrices predictivas para el modelo $F=ma$:
\begin{verbatim}
A_p = -0.5 * (dt^3 / m);
A_v = - (dt^2 / m);
A_a = - (dt / m);
H = 2 * (A' * W * A);
f = 2 * (A' * W * B);
\end{verbatim}

\bibliographystyle{IEEEtran}
\bibliography{references}

\begin{IEEEbiography}{Jorge Apud}
Estudiante de Ingeniería Eléctrica, PUC Chile. Intereses en control automático y optimización de redes.
\end{IEEEbiography}

\begin{IEEEbiography}{Patricio Hinostroza}
Estudiante de Magíster en Ciencias de la Ingeniería, PUC Chile. Intereses en sistemas ciberfísicos y robótica.
\end{IEEEbiography}

\end{document}